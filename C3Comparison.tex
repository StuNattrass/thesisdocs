

\section{Introduction}
No single species can allocate unlimited resources to each life history trait \cite{law1979optimal}. Instead, species life histories are determined by the allocation of limited resources to different areas of need. In plant species, the three most important life history traits are reproduction, growth from seedling to adult, and defence against both herbivory and abiotic factors such as fire (\cite{bazzaz1987allocating}). This leads to trade-offs between life history traits; for example, a species with high levels of resource allocated to increasing fecundity will not be able produce seeds of a large mass (\cite{turnbull1999seed}) and will therefore experience a decrease in its juvenile growth rate (\cite{gross1984effects}), or a species with rapid growth may be more susceptible to damage by storm winds or other mortality pressures such as large scale herbivory (e.g. \cite{wright2010functional,fine2006growth}). Theory has shown that these trade-offs can allow two or more species to coexist while competing for  the same resources in an environment (e.g. \cite{kisdi2003coexistence,levins1971regional,bonsall2004life}), suggesting that trade-offs are important for sustaining high levels of biodiversity in nature.

However, conditions under which species coexist alongside others will also be dependent on other, abiotic factors. Several previous studies have suggested that disturbance events also play an important role in promoting and maintaining diversity (e.g. \cite{sousa1984role,denslow1987tropical}).  Recent theoretical work (\cite{miller2011frequency}) show that different measures of disturbance, such as frequency or intensity, will have very different effects, even when the total biomass lost to disturbance over a given time period is taken into account. While many empirical studies consider disturbance as a single parameter (e.g. \cite{molino2001tree,peterson1997tornado,nakagawa2000impact}), some studies do demonstrate that different factors determining disturbance can affect the community structure differently. Hall and Miller \cite{hall2012diversity} use bacterial populations to demonstrate that the frequency and intensity of disturbance events have different impacts, while Denslow \cite{denslow1980patterns} indicates that communities with large, infrequent disturbances may be more diverse than those where disturbance events are more frequent, yet clear a smaller area (e.g. tree-fall gaps).

Here, all possible trade-offs between the three most important plant life history traits - reproduction, juvenile growth and defence - are considered. Defence is considered as the ability of an individual to withstand a disturbance event, that is, an event that results in the death of large number of individuals and alters niche opportunities within the community, while seed production (seed number per capita per year) is used as the measure of resource allocation. We therefore consider 8 models. The model where all species are identical gives neutral dynamics, and the coexistence of the two species is governed by chance, while the 3 where the species differ in a single trait (no trade-off) demonstrate competitive exclusion of the weaker species. When trade-offs are between two or more traits, the effects of system capacity and varied disturbance regimes  are considered, and we show that trade-offs between fecundity and growth, or growth and defence, can support two species for large system sizes, as can a three dimensional trade-off, although a fecundity-defence trade-off cannot. The probability of species disturbance resistance parameters resulting in coexistence is calculated, and we demonstrate that a three dimensional trade-off gives the greatest likelihood of coexistence, an order of magnitude larger than that of a growth-defence trade-off. Species specific disturbance intensities are then linked to allow comparison between the three dimensional trade-off and a fecundity-growth trade-off where disturbance affects all species equally, and we show that if the more fecund of the two species has a slight advantage in defence against disturbance, the likelihood of coexistence is maximised.


\section{Models}
Here, we build on the model presented in Chapter 2, introducing a single extra factor. Now, we allow species to differ in the level of resource they dedicate towards resistance to disturbance events, so an individual of species $i$ will experience a species specific probability of death during a disturbance $I_i$. The model is described in a non disturbance time  step by the following transition probabilities (where $n=N_1(t)$ and $N-n=N_2(t)$);
\begin{align}
\label{inc}P(\text{increase}(n))=&\frac{N-n}{N}\frac{s_1 n}{s_1 n +s_2(N-n-1)}\exp \left( -s_2 \frac{N-n-1}{N} x\right),\\
\label{dec}P(\text{decrease}(n))=&\frac{n}{N}\frac{s_2 (N-n)}{s_1 (n-1) +s_2(N-n)}  \\
& + \frac{n}{N}\frac{s_1 (n-1)}{s_1 (n-1) +s_2(N-n)}\left(1-\exp \left( -s_2 \frac{N-n}{N} x\right)\right), \notag \\
\label{stay}P(\text{stay}(n))=&1-P(\text{increase}(n))-P(\text{decrease}(n)).
\end{align}
where $s_i$ is the per capita annual seed production of species $i$, $N$ is the system capacity - the maximum number of adults that can be sustained by the environment, and $x=C(1/g_1 - 1/g_2)/2$ is a measure of growth rate differences (with $C$ canopy height and $g_i$ the sapling growth rate of species $i$).

During a disturbance event, each individual of species $i$ will die with probability $I_i$. If there are $d_i$ deaths of species $i$, then the total deaths is given by $d=\sum_i d_i$. Note that $d$ is now dependent on the species composition of the community when the disturbance strikes, as well as the inherent properties of the disturbance. The expected number of deaths is given by
\begin{equation}
\label{avdeaths} \bar{d}=\sum_i I_i N_i(t)
\end{equation}
where $N_i(t)$ is the population of species $i$ at time $t$. Once all $d$ deaths occur, the remaining $N-d$ individuals compete for the opened sites in the system. \textbf{If the populations of the two species are given by $n_1^*,n_2^*$ in the immediate aftermath of a disturbance, the probability of each gaps being successfully colonised by species one is given by}
\begin{equation}
\label{sp1}
Sp_1(n_1^*,n_2^*)=\frac{s_1 n_1^*}{s_1n_1^*+s_2n_2^*}\exp \left(-s_2 x\frac{n_2^*}{N}\right).
\end{equation}

\textbf{We assess coexistence by considering invasion analysis. The average or expected change in a time step is approximated by
\begin{align}
\label{ac}
\text{AverageChange}(n)=&(1-f)\left(P(\text{increase}(n))-P(\text{decrease}(n))\right)  \\
&+f\left(-nI_1 +(nI_1+(N-n)I_2)Sp_1(n(1-I_1),(N-n)(1-I_2))\right).\notag
\end{align}
Appendix~\ref{appapproximations} shows that for each of the models considered, this approximation improves with system capacity $N$, and fits the actual change well. Coexistence occurs when on average, both species increase when rare; that is when}
\begin{align*}
\text{AverageChange}(1)&>0,\\
\text{AverageChange}(N-1)&<0. \end{align*}

We can now consider a collection of models by varying a different combination of the three traits; fecundity, juvenile growth rate and defence or resistance to disturbance. This gives a selection of 8 models, one where the species are identical, 3 where they differ in a single parameter, 3 where they differ in two ways and the third parameter is equal, and a final model where species can differ in all three ways. The first four of these are trivial to analyse, but the models where species differ in at least two of the life history parameters give more interesting dynamics.

\begin{center}
\begin{tabular}{|c|l|c|} \hline
\multicolumn{3}{|c|}{Table of parameters} \\ \hline
Parameter & Description & Default value \\ \hline
$s_1$ & per capita annual seed production of species 1 & 500 \\ \hline
$s_2$ & per capita annual seed production of species 2 & 50\\ \hline
$g_1$&growth rate of species 1 juveniles & 13 \\
&(measured in mm/yr for diameter at breast height) &\\ \hline
$g_2$&growth rate of species 2 juveniles & 13.21 \\
&(measured in mm/yr for diameter at breast height) &\\ \hline
$C$& size of individuals at canopy height & 100 \\
& (measured in mm for diameter at breast height&\\ \hline
$x$&$C(1/g_1-1/g_2)/2$&0.06\\
&time window for successful secondary colonisation by species 2&\\ \hline
$N$ & system capacity: number of individuals the region can support & 1000 \\ \hline
$n$ & number of species 1 individuals & N/A \\ \hline
$T_D$& average time between disturbances in yrs & 10 \\ \hline
$f$& within event disturbance probability & $100/(T_D N)$ \\ \hline
$I_1$& intensity of disturbance for species 1 & N/A \\ \hline
$I_2$& intensity of disturbance for species 2 & N/A \\ \hline
$y$ &Parameter linking species intensities; given by $I_1/I_2$ & N/A \\ \hline
\end{tabular} \end{center}

\section{results}
When all species are identical, the model is simply a one-dimensional random walk, where disturbance events of intensity $I$ merely increase the distance it is possible to travel in a single step. Coexistence in this model is merely a function of chance. If the species merely differ in a single parameter, then that with the higher growth rate or fecundity, or with the lower mortality in a disturbance regime will competitively exclude the other.

\subsection{Fecundity-growth trade-off}
The case where species differ in both fecundity and growth ($I=I_1=I_2$) is analysed in detail in Chapter 2. In summary, for sufficiently low growth rate differentials $x$,  the more fecund species 1 can competitively exclude the rapidly growing species 2, both in the presence of disturbance and in a temporally homogeneous environment. For intermediate $x$, coexistence is possible in the homogeneous environment, and this behaviour persists for low intensity or low frequency disturbances. As disturbance intensity increases (providing the frequency $f$ is greater than the expected time to extinction $t_{extinct}$), the fecundity advantage of species 1 will lead to it competitively excluding species 2. Thus, disturbance can lead to a decrease in diversity. When $x>x_{max}$ is large, species 2 will exclude the more fecund species 1 due to it's superior growth rate. In this case, disturbance has a more complex effect. For frequencies $f$ such that $1/f > t_{extinct}$, intermediate disturbances will lead to coexistence of two species, while high intensities will allow the previously excluded species 1 to claim all sites and exclude species 2. Low intensities are insufficient to change the temporally homogeneous diversity, although may extend the time to coexistence. When $1/f < t_{extinct}$, species 2 will persist in monoculture. System size impacts the region of coexistence by increasing $t_{extinct}$, extending the region of more complex behaviour.

For a large system, or disturbances more frequent that $t_{extinct}$, the range of intensities that can give coexistence varies with the growth rate differential $x$, and is maximised at $x=x_{max}$, and also with the difference in fecundities. As seed numbers are varied, the maximum range of disturbance is at the point $s_1=s_2\exp(s_2x)$. That is, the range is maximised when the function $I=1-\ln(s_1/s_2)/(s_2x)$, formed by setting $\text{AverageChange}(1)=0$, returns $I=0$.

\subsection{Fecundity-defence trade-off}
When species have identical growth rates ($x=0$), yet differ in disturbance response and fecundities $s_i$, it is possible to find analytically the range of $I_1 - I_2$ space that gives coexistence for given fecundities $s_i$. Setting $\text{AverageChange}(1)=0$ and solving for $I_1$ gives
\begin{align}
\label{fdac1sol}
I_1=&\frac{I_2(f(N-1)s_1-A(1)s_2(N-1)) +A(1)(s_1+s_2(N-1))}{I_2f(N-1)(s_1-s_2) +fs_2(N-1) +A(1)s_1}\notag \\
=&\frac{\alpha_1 I_2 +\beta_1}{\gamma_1 I_2+\delta_1}
\end{align}
while setting $\text{AverageChange}(N-1)=0$ gives the curve
\begin{align}
\label{fdacnsol}
I_1=&\frac{I_2(f(N-1)s_1-A(N-1)s_2) +A(N-1)(s_1(N-1)+s_2)}{I_2f(N-1)(s_1-s_2) +fs_2(N-1) +A(N-1)s_1(N-1)}\notag \\
=&\frac{\alpha_{N-1} I_2 +\beta_{N-1}}{\gamma_{N-1} I_2+\delta_{N-1}}
\end{align}
where $A(n)=(1-f)(P(\text{increase}(n))-P(\text{decrease}(n)))$ is constant in $I_2$. The region of coexistence is therefore the area between the two curves, as demonstrated in Figure~\ref{fd}(a). By noting that the indefinite integral of $(a x +b)/(c x +d) = ax/c+(bc-ad)\ln(d+cx)/c^2$, we can write a solution for the size of the region of $I-$space that can give coexistence;
\begin{align}
&\min \left(1, \int_0^1 dI_2 \frac{\alpha_1 I_2 +\beta_1}{\gamma_1 I_2+\delta_1} \right) - \min \left(1, \int_0^1 dI_2 \frac{\alpha_{N-1} I_2 +\beta_{N-1}}{\gamma_{N-1} I_2+\delta_{N-1}} \right) \notag \\
=&\min\left(1,\frac{\alpha_1}{\gamma_1}+\frac{(\beta_1 \gamma_1 - \alpha_1 \delta_1)(\ln(\delta_1 +\gamma_1)-\ln(\delta_1)}{\gamma_1^2}\right) \\
&-\min \left(1,\frac{\alpha_{N-1}}{\gamma_{N-1}}-\frac{(\beta_{N-1} \gamma_{N-1} - \alpha_{N-1} \delta_{N-1})(\ln(\delta_{N-1} +\gamma_{N-1})-\ln(\delta_{N-1})}{\gamma_{N-1}^2}\right). \notag
\end{align}
The minimum is taken because for sufficiently large $s_1$, the function given by \eqref{fdac1sol} is greater than one for all $I_2 \in (0,1)$, but an intensity of greater than 1 is impossible. Hence, the region of coexistence does not change with $s_1$ in a smooth manner, as shown in Figure~\ref{fd}(d). The peak range of coexistence is given by intermediate $s_1$ (for fixed $s_2=50$), with the region tending to zero as the difference in species fecundities tends to infinity. For the chosen parameters, with system size $N=1000$ the peak probability of coexistence (when intensities for the two species are chosen at random) is approximately 0.120.
\begin{figure}[htbp]
\begin{tabular}{cccc}
(a)&&(b)&\\
&\includegraphics[width=2in]{fdtotdis10.pdf}&&\includegraphics[width=2in]{fdtointwN.pdf} \\
(c)&&(d)&\\
&\includegraphics[width=2in]{fdtointwTd.pdf}&&\includegraphics[width=2in]{fdtointws1.pdf}
\end{tabular}
\caption{\textbf{Fecundity-defence trade-off:} (a) An example of the region of coexistence. The blue area (top and left) is where species 1 can invade when rare, while the pink area (bottom and right) gives the region where species 2 can invade. The region where they overlap is the region that gives coexistence with $N=100$. (b) The effects of changing system capacity on the size of the region of $I-$space that predicts coexistence. The region of coexistence peaks at intermediate system capacity $N\approx 850$, and declines with system size above this value. (c) The change in the likelihood of coexistence as time between disturbances $T_D$ is increased. The size of this region is maximised at intermediate values of $T_D \approx 10$, and tends to 0 as $T_D$ goes to infinity. (d) The change in the likelihood of coexistence as $s_1$ is increased. Note the discontinuity where \eqref{fdac1sol} becomes greater than one as $s_1=14225$.  Parameters $s_1=500,s_2=50,N=1000,T_D=10$ unless specified.}
\label{fd}
\end{figure}

Increasing frequency, reducing the time between disturbances $T_D$, also has a dramatic effect on the results. For very low frequency, no defence response combination is unable to overcome the fecundity disadvantage and allow species 2 to survive. As frequency is increased, then very high $I_1$ can combine with low $I_2$ to give a small region of coexistence. For example, if a species devoted a great deal of resource to surviving fire, it could persist with a species very susceptible to fire but with a greater fecundity to spread after the disturbance. The range of parameters $I_i$ that give coexistence in this way increases with frequency until a threshold is reached (see Figure~\ref{fd}(c)) at which point, it is possible for species 2 to outcompete species 1 if $I_2<<I_1$. The region of $I-$ space then begins to decline in area and move towards the curve $I_1=I_2s_1/(I_2s_1+s_2(1-I_2))$. In the limit as $f \to 1$ where `disturbance' events are so frequent as to provide an homogeneous environment themselves, both functions given by \eqref{fdac1sol} and \eqref{fdacnsol} reach this curve, and coexistence is not possible. If $I_1>I_2s_1/(I_2s_1+s_2(1-I_2))$ then species 2 will dominate the environment, while if $I_1<I_2s_1/(I_2s_1+s_2(1-I_2))$ species 1 will exclude the less fecund species 2.

As system capacity $N$ increases, the range of coexistence shows a peak at $N \approx 850$. For wood or forest sized systems above this value, the region will decline rapidly with increased system capacity. When the system capacity tends to infinite, the region of $I-$space giving coexistence tends to zero. A fecundity-defence trade-off cannot sustain two species in a large system, and has little effect in assisting biodiversity when system size is that or a forest or wood. \textbf{The maximal effect on coexistence is restricted to a very small range of parameters, with steep declines in the probability of coexistence when moving away from these optimal parameter values.} 

\subsection{Growth-defence trade-off}
A growth-defence trade-off ($s=s_1=s_2=50$) responds to changes in system capacity and frequency in a very different manner to the fecundity-defence trade-off outlined above. Here frequency has little effect providing $1/f<t_{extinct}$, while if this condition is not satisfied, the faster growing species will exclude its competitor for any disturbance intensity regime. The response to system size is similar to that of the fecundity-growth trade-off, where as system size increases, \textbf{where the system asymptotes to a fixed region where coexistence is predicted. Setting $\text{AverageChange}(1)=0$ and $\text{AverageChange}(N-1)=0$ gives $I_1$ as quadratic functions of $I_2$, $LBR(I_2), UBR(I_2)$, outlined in Appendix~\ref{approots}}.


For the wide range of $x_s$ considered here, the roots where the positive square root is taken are always above one, and therefore outside the possible parameter space for disturbance intensity, while the negative square roots are the solutions that helps to determine the system behaviour.
\begin{figure}[htbp]
\begin{tabular}{cccc}
(a)&&(b)&\\
&\includegraphics[width=2in]{gdx26}&&\includegraphics[width=2in]{gdtointwN} \\
(c)&&(d)&\\
&\includegraphics[width=2in]{gdtointwTd}&&\includegraphics[width=2in]{gdtointwx}
\end{tabular}
\caption{\textbf{Growth-defence trade-off:} (a) An example scenario, where coexistence can occur if both species suffer high mortality in a disturbance event. (b) The effects of changing system capacity $N$ on the likelihood of coexistence. As $N$ increases, the probability of coexistence stabilises as the integral asymptotes to a fixed value. (c) The change in the likelihood of coexistence as time between disturbances $T_D$ is increased. The size of this region is maximised at intermediate values of $T_D \approx 45$, and tends to 0 as $T_D$ goes to infinity. (d) The change in the likelihood of coexistence as $x$ is increased. Note the discontinuity where  the size of the region of coexistence peaks at $x\approx 0.2$. Parameters $s=50,x=0.06,N=1000,T_D=10$ unless specified.}
\label{gd}
\end{figure}

We can numerically integrate the solutions in Appendix~\ref{approots} to find the area of the region of coexistence, using the following:
\begin{equation}
\int_{\max(0,I^*)}^1 dI_2 LBR(I_2) - \int_{\max(0,I^*)}^1 dI_2 UBR(I_2),
\end{equation}
where $I^*$ is the point at which $LBR(I^*)=UBR(I^*)$. This numerical integration allows a study the behaviour of the system as $x$ is varied. When $x=0$ and the growth rates are identical, the species with the most resistance to disturbance (lowest $I_i$) will exclude its competitor.  As $x$ is increased, four distinct regions will occur, as in Figure~\ref{gd}(a). When both species display high resistance to disturbance (low $I_i$) there is a small region where neither $\text{AverageChange}(1)>0$ or $\text{AverageChange}(N-1)<0$ are satisfied, and founder control occurs, where coexistence does not occur and the successful species depends on the initial populations, subject to stochastic noise. When both species have higher intensities, there exists a region where coexistence is expected. Together the two regions of coexistence and founder control form a band from across $I-$space that separate regions where species and species 2 will exclude the other. We find the region of coexistence peaks at intermediate $x \approx 0.2$, when the probability of two species with randomised defence regimes coexisting is approximately $0.15$. As $x$ increases beyond this, the region of coexistence declines and tend to the region of $I-$space where $I_2>>I_1$, eventually tending to zero as $x$ tends to infinity.

\subsection{Three dimensional trade-off}
When species are allowed to vary in all three traits, fecundity, growth and defence, the region of coexistence is given by integrating the functions determined in Appendix~\ref{approots}. For the three dimensional trade-off, this integral takes the form
\begin{equation}
\int_{\max(0,I^*,I^+)}^1 dI_2 LBR(I_2) - \int_{\max(0,I^*,I^{++})}^1 dI_2 UBR(I_2),
\end{equation}
where $LBR(I^*)=UBR(I^*)$, $LBR(I^+)=0$, and $UBR(I^{++})=0$. As in the growth-defence case, the roots with the negative square root are those that control the behaviour of the system. Numerically integrating, we can again calculate the region of coexistence in $I-$space. The region of coexistence in $I-$space is shown in Figure~\ref{full}(a). Calculating the areas of coexistence shows that for fixed seed numbers, the size of the area of coexistence will peak at intermediate values of $x$, while for fixed $x$, the range of coexistence will increase as the discrepancy in fecundities becomes more pronounced. This increase will asymptote to the point where it is not possible for species 2 to exclude species 1, but where a significant proportional of trait space will give coexistence (See Figure~\ref{full}(d,f)). The region of coexistence generated by this model is consistently larger than that of the other two models (with $s_1=500,s_2=50,T_D=10,N=1000$, we see that the maximum range of coexistence is given when $x\approx 0.22$, and this region of coexistence has area $\approx 0.271$)
\begin{figure}[htbp]
\centering
\begin{tabular}{rrrr}
(a)&&(b)&\\
&\includegraphics[width=2in]{fullexample}&&\includegraphics[width=2in]{fullintwithN} \\
(c)&&(d)&\\
&\includegraphics[width=2in]{fullintwTd}&&\includegraphics[width=2in]{fullintwiths1} \\
(e)&&(f)&\\
&\includegraphics[width=2in]{fulltointwx}&&\includegraphics[width=2in]{fulllarges1}
\end{tabular}
\caption{\textbf{Three-dimensional trade-off:} (a) An example of $I-$space. The pink region to the bottom and right are where species 2 is dominant, while the blue region (top and left) give species 1 monoculture. Where these regions overlap there is coexistence. (b)  (c) The change in the region of coexistence as $x$ varies, with a peak at intermediate values. (d) The size of the region given in (b) asymptotes as $s_1$ increases. Parameters $s_1=500,s_2=50,x=0.06,N=1000,T_D=10$ used to generate images unless specified. (e) The size of the region of coexistence and how it varies with $x$. The probability of coexistence increases rapidly with low $x$, and then experiences a slower increase for $0.046<x<0.22$. Above $x=0.22$ (f) When $s_1$ tends to infinity, a large proportion of $I-$space can support both species, although species 2 dominance is not possible ($s_1=10^6$).}
\label{full}
\end{figure}

\subsection{Linked intensities}
However, it is perhaps unrealistic to allow the responses to disturbance to move at random through $I-$space. Species specific responses are likely to be linked, such that as one increases, the other also increases. To simulate this, we consider the case where $I_1=yI_2$, such that $y$ is a measure of the differences in the life history strategies of the two species, and $I_2$ is the force of a given disturbance event, normalised to the response of species 2. Note that for $y\neq 1$ one species will experience certain mortality while the other may retain some individuals. As intensity increases beyond this point, the two intensities will converge at one, but as the species with the lower resistance is already doomed, this will not affect the likelihood of species coexisting.

In the three dimensional trade-off model, we consider the effects of changing $y$ on the range of disturbances that can give coexistence for the parameters $s_1=500,s_2=50,x=0.06,T_D=0.2$.

We can then plot the average change at the boundary as a function of $I_2$ for differing $y$. We find dramatically different behaviour as $y$ varies. For $y$ sufficiently large ($y>1.39$ for the current parameters), such that species 2 has a huge advantage in survival of a disturbance event, we find that for all intensities, coexistence does not occur, and species 2 exists in monoculture. Here, the increased fecundity of species 1 is not sufficient to overcome the dual advantage of species 2, with its superior juvenile growth rate and increased resistance to disturbance, as shown in Figure~\ref{largey}.
\begin{figure}[htbp]
\includegraphics[width=4.5in]{highy}
\caption{Plots of $\text{AverageChange}(1)$ in blue and $\text{AverageChange}(N-1)$ in pink. $\text{AverageChange}(N-1)<1$ is satisfied for all intensities, but $\text{AverageChange}(1)>0$ never holds, so species 2 will exist in monoculture. $\text{AverageChange}(1)>\text{AverageChange}(N-1)$ for all $I_2$.}
\label{largey}
\end{figure}
As $y$ decreases, the behaviour then changes, as the curve for $\text{AverageChange}(1)>0$ is satisfied for a range of intermediate disturbance intensities. At the same time, $\text{AverageChange}(N-1)$ remains below zero for all intensities $I_2$, meaning that coexistence is possible for intermediate intensities, but either side of this range species 2 will competitively exclude species 1. For the given parameters, the range of $y$ exhibiting this behaviour is approximately $1.16 \leq y \leq 1.38$. Decreasing $y$ further, making the two species response to disturbance more similar, we get yet more behaviour. For $y\leq 1.15$, $\text{AverageChange}(N-1)$ has two real roots, and is positive between these roots. While $y>1$, both these roots occur in the interval $[0,1)$, as do two roots of $\text{AverageChange}(1)$. In this case, there are four distinct responses to disturbance as it increases in intensity (see Figure~\ref{fourbits}). At low intensity, disturbance events are not strong enough to promote the more fecund species 1, so species 2 exists in monoculture. As intensity increases, then $\text{AverageChange}(1)>0$ is satisfied, while $\text{AverageChange}(N-1)<0$ continues to hold, giving coexistence of both species. Further increasing intensity, however, will result in $\text{AverageChange}(N-1)$ becoming positive. Here, species 2 will be competitively excluded, and species 1 will monopolise the system. Increasing intensity yet further results in $\text{AverageChange}(N-1)$ dropping back below zero, to give a secondary region where our invasion analysis predicts coexistence. However, for the parameters studied here, the intensity here is high, and leaves a small remaining population (when $N=1000$, the remaining population is, on average, approximately 30 individuals). This reduced population is very susceptible to the stochasticity in the model, and all simulations result in one species going extinct by chance, although the identity of the surviving species varies between simulations. Increasing intensity even more, we have that $\text{AverageChange}(1)$ also drops below zero, so the theory predicts species two monoculture, which is once again matched by simulations, even given the extremely high intensities involved.

Once $y\leq 1$ there is only one root of $\text{AverageChange}(1)$ and one root of $\text{AverageChange}(N-1)$ in the interval $(0,1)$ (See Figure~\ref{onerooteach}). We therefore see a different type of behaviour again, with species 2 excluding the more fecund species 1 at low intensities, coexistence occurring at intermediate intensities, and species 1 excluding species 2 as high intensities.

\begin{figure}[htbp]
\includegraphics[width=4.5in]{fourbits}
\caption{Plots of $\text{AverageChange}(1)$ in blue and $\text{AverageChange}(N-1)$ in pink. Both curves have two real roots in the interval $[0,1]$, resulting in two bands of coexistence predicted by theory, with alternating monocultures surrounding them, species 2 monoculture at low and very high intensities, and species 1 at intermediate intensities. Note the higher band of predicted coexistence is not conformed by simulations, instead experiencing extinction of a random species due to the high intensity of disturbance events. $\text{AverageChange}(1)>\text{AverageChange}(N-1)$ for all $I_2$.}
\label{fourbits}
\end{figure}
\begin{figure}[h]
\includegraphics[width=4.5in]{onerooteach}
\caption{Plots of $\text{AverageChange}(1)$ in blue and $\text{AverageChange}(N-1)$ in pink. Both curves have a single real root in the interval $[0,1]$, resulting in a band of coexistence predicted by theory at intermediate intensity, with species 2 dominating at lower intensities while species 1 dominates at high intensities. $\text{AverageChange}(1)>\text{AverageChange}(N-1)$ for all $I_2$.}
\label{onerooteach}
\end{figure}
As $y$ decreases towards zero, the range of intensities that can support both species will tend to a small interval, of width approximately $0.005$, at very low intensities ($0.005\lessapprox y \lessapprox 0.01$). We find that it is possible only to slightly increase the range of intensities giving coexistence from that of the case $y=1$, with $y=0.90$ (2 significant figures) giving a small increase in intensity range leading to coexistence ($0.4666>0.4663$), but this value declining again for $y$ below this point. That is, when the more fecund species 1 has a slight advantage in resistance to disturbance, the range of disturbances that the system can survive while maintaining it's full biodiversity is maximised.

\section{Discussion}
Both life history trade-offs and disturbance events have been suggested as mechanisms that can promote and support high levels of diversity in nature. Here, we demonstrate that trade-offs among the three major life history traits for plants can give coexistence of at least two species competing within the same niche, and numerically quantify the likelihood of coexistence for each trade-off. However, the effectiveness of these trade-offs in supporting multiple species varies dramatically. Where species no not differ in juvenile growth rates, only small communities can possibly support more than a single species, and even in these small communities coexistence is extremely unlikely outside a very narrow range of parameters. This result suggests that a trade-off between seed production and resilience cannot realistically contribute to the diversity of a community, a view supported by the lack of empirical evidence in favour of this trade-off. \textbf{Using data from \cite{martin2010dispersal}, \cite{martin2010divergence} concluded that ``it is unlikely that... survivorship come[s] at the expense of fecundity.'' While some other studies find a correlation between (e.g. \cite{marquis1984leaf,gwynn2005resistance}) this relationship is often weak, suggesting that the selection pressure for such a trade-off is weak. We propose that evidence for this trade-off is a consequence of other, more strongly selected trade-offs such as the growth-survival and fecundity-growth trade-offs.}

Meanwhile, trade-offs between growth and either fecundity or disturbance (or both) resistance can support multiple species for any system capacity $N$. In a large system, we demonstrate that a trade-off involving all three of the major plant traits will give a probability of randomly selected disturbance resistances supporting two species higher than the probability given by a growth-defence trade-off alone. However, when species responses to disturbance are proportional (such that disturbance intensity for one species is doubled, the intensity for the second species is also doubled), the three dimensional trade-off does not significantly improve the likelihood of coexistence when compared to a fecundity-growth trade-off. These results suggest that the trade-off between fecundity and juvenile growth rate, or competition and colonisation, contributes much more to the maintaining of high biodiversity than trade-offs involving disturbance resistance.  


We conclude that a trade-off between fecundity, in the form of per capita annual seed production, and juvenile growth rates is more significant in sustaining biodiversity than trade-offs between growth and defence or fecundity and defence, and while species specific reactions to disturbance can slightly improve the likelihood of a fecundity-growth trade-off allowing two species coexistence, this increase is not significant. This concurs with empirical studies, which have found a great deal of support for a trade-off between fecundity and growth rate, or the equivalent competition-colonisation trade-off (e.g. \cite{levins1971regional,yu2001competition,tilman1994competition,adler2000space}). The high level of occurrence for this trade-off indicates that it has been an important driver in the evolution of those diverse communities, allowing two or more species to effectively occupy the same resource niche by the different allocation of that resource to their life history traits. That a trade-off between growth and defence can support some coexistence suggests that some support should be found for this trade-off in empirical studies, and this is indeed the case (e.g. \cite{wright2010functional,fine2006growth}). However, support for this is much less widespread than the fecundity-growth or competition-colonisation trade-off, which further supports our conclusion that the latter is the most significant driver of biodiversity.

\section*{Appendices}
 \appendix
 \numberwithin{equation}{section}
 \section{Approximating the expected change function}
 \label{appapproximations}
 The expected change for the model when at the lower boundary $n=1$ is given by
 \begin{align}
\label{acreal1difi}&\text{AverageChangeReal}(1)=\frac{-I_1+(N-1)I_2^{N-1}(1-I_1)}{dNT_D}\\
&+\frac{(1-I_1)\sum_{k=1}^{N-2}\sum_{j=k}^{N-2}k I_2^j(1-I_2)^{N-2-j}{N-2\choose j} Sp_1(1,N-1-j)^k(1-Sp_1(1,N-1-j))^{j-k} {j \choose k}}{dNT_D} \notag \\
 &+ \left(1-\frac{1}{dNT_D}\right)(P(\text{Increase}(1))-P(\text{Decrease}(1))), \notag
 \end{align}
 while at the upper boundary $n=N-1$, the expected change is given by
\begin{align}
 \label{acrealtotdifi}&\text{AverageChangeReal}(N-1)= \frac{-I_1^{N-1}(N-1) +I_2(1-I_1^{N-1})}{dNT_D} \\
 &-\frac{(1-I_2)\sum_{k=1}^{N-2}\sum_{j=k}^{N-2}k I_1^j(1-I_1)^{N-2-j}{N-2\choose j} Sp_1(N-1-j,1)^{j-k}(1-Sp_1(N-1-j,1))^{k} {j \choose k}}{dNT_D} \notag \\
 &+ \left(1-\frac{1}{dNT_D}\right)(P(\text{Increase}(N-1))-P(\text{Decrease}(N-1))). \notag
  \end{align} 
These are approximated using \eqref{ac}. Figure~\ref{fig:fdtoapprox} shows how the error between the actual expected change and the approximation declines with system capacity $N$ for a trade-off between fecundity and defence ($x=0$), while Figures~\ref{fig:gdtoapprox}~and~\ref{fig:fullapprox} show that this result holds for the growth-defence trade-off ($s_1=s_2$) and the three dimensional trade-off respectively. The error was calculated for intensities $I=0.1,0.2,0.3,...,0.9$ and time between disturbances $\ln(T_D)=1,2,3,...,8,$, therefore giving 648 error values for each system size. The approximation is accurate even at relatively small $N>400$. While the results from only one set of parameters are shown, these results are robust to parameter changes.
 \begin{figure}[th]
\centering
   \begin{tabular}{rrrr}
   (a)&&(b)&\\
  &\includegraphics[width=2.5in]{FDmeanerr.pdf} && \includegraphics[width=2.5in]{FDmaxerr.pdf} \\
  (c)&&(d)&\\
  &\includegraphics[width=2.5in]{FDtmeanerr.pdf} && \includegraphics[width=2.5in]{FDtmaxerr.pdf} \end{tabular}
  \label{fig:fdtoapprox}
   \caption{\textbf{Fecundity-defence trade-off:} (a-b)  How the mean (a) and maximum (b) error size at the lower boundary decrease with increased system capacity $N$. (c-d) How the mean (c) and maximum (d) error size at the upper boundary decrease with increased system capacity $N$. The error was calculated for intensities $I_1,I_2=0.1,0.2,0.3,...,0.9$ and time between disturbances $\ln(T_D)=1,2,3,...,8.$ Parameters used are $s_1=500,s_2=50$.}
   \end{figure}
    \begin{figure}[th]
\centering
   \begin{tabular}{rrrr}
   (a)&&(b)&\\
  &\includegraphics[width=2.5in]{GDmeanerr.pdf} && \includegraphics[width=2.5in]{GDmaxerr.pdf} \\
  (c)&&(d)&\\
  &\includegraphics[width=2.5in]{GDtmeanerr.pdf} && \includegraphics[width=2.5in]{GDtmaxerr.pdf} \end{tabular}
  \label{{fig:gdtoapprox}}
   \caption{\textbf{Growth-defence trade-off:} (a-b)  How the mean (a) and maximum (b) error size at the lower boundary decrease with increased system capacity $N$. (c-d) How the mean (c) and maximum (d) error size at the upper boundary decrease with increased system capacity $N$. The error was calculated for intensities $I_1,I_2=0.1,0.2,0.3,...,0.9$ and time between disturbances $\ln(T_D)=1,2,3,...,8.$ Parameters used are $s_1=50,s_2=50,x=0.06$.}
    \end{figure}
    \begin{figure}[th]
\centering
   \begin{tabular}{rrrr}
   (a)&&(b)&\\
  &\includegraphics[width=2.5in]{FDGmearerr.pdf} && \includegraphics[width=2.5in]{FDGmaxerr.pdf} \\
  (c)&&(d)&\\
  &\includegraphics[width=2.5in]{FDGtmearerr.pdf} && \includegraphics[width=2.5in]{FDGtmaxerr.pdf} \end{tabular}
  \label{fig:fullapprox}
   \caption{\textbf{Fecundity-growth-defence trade-off:} (a-b)  How the mean (a) and maximum (b) error size at the lower boundary decrease with increased system capacity $N$. (c-d) How the mean (c) and maximum (d) error size at the upper boundary decrease with increased system capacity $N$. The error was calculated for intensities $I_1,I_2=0.1,0.2,0.3,...,0.9$ and time between disturbances $\ln(T_D)=1,2,3,...,8.$ Parameters used are $s_1=500,s_2=50,x=0.06$.}
    \end{figure}
 
 \section{Roots of average change for $x>0$}
 Using Mathematica 8.0.1.0 to set $\text{AverageChange}(1)=0$ and rearranging to solve for $I_1$ gives the solution
 \begin{align*}
 LBR(I_2)=&\frac{0.5 \left(100 e^{\frac{s_2 (N-2) x}{N}+\frac{(1-I_2) s_2 (N-1) x}{N}} s_2^2   N^3 -100 e^{\frac{s_2 (N-2) x}{N}+\frac{(1-I_2) s_2 (N-1) x}{N}} I_2 s_2^2   N^3+100 e^{\frac{s_2 (N-2) x}{N}} I_2   s_1 s_2 N^3+100 e^{\frac{s_2 (N-2)   x}{N}} I_2 s_1^2 N^2-300 e^{\frac{s_2   (N-2) x}{N}+\frac{(1-I_2) s_2   (N-1) x}{N}} s_2^2 N^2+300   e^{\frac{s_2 (N-2) x}{N}+\frac{(1-I_2)   s_2 (N-1) x}{N}} I_2 s_2^2   N^2-100 e^{\frac{s_2 (N-2) x}{N}} s_1   s_2 N^2+200 e^{\frac{s_2 (N-2)   x}{N}+\frac{(1-I_2) s_2 (N-1)   x}{N}} s_1 s_2 N^2-300 e^{\frac{s_2   (N-2) x}{N}} I_2 s_1 s_2   N^2-100 e^{\frac{s_2 (N-2) x}{N}+\frac{(1-I_2) s_2 (N-1) x}{N}} I_2 s_1   s_2 N^2+e^{\frac{(1-I_2) s_2 (N-1)   x}{N}} s_1^2 T_D N^2-1 e^{\frac{s_2   (N-2) x}{N}+\frac{(1-I_2) s_2   (N-1) x}{N}} s_1 s_2 T_D   N^2-100 e^{\frac{s_2 (N-2) x}{N}}   s_1^2 N-100 e^{\frac{(1-I_2) s_2   (N-1) x}{N}} s_1^2 N+100 e^{\frac{s_2   (N-2) x}{N}+\frac{(1-I_2) s_2   (N-1) x}{N}} s_1^2 N-100 e^{\frac{s_2   (N-2) x}{N}} I_2 s_1^2 N+200   e^{\frac{s_2 (N-2) x}{N}+\frac{(1-I_2)   s_2 (N-1) x}{N}} s_2^2 N-200   e^{\frac{s_2 (N-2) x}{N}+\frac{(1-I_2)   s_2 (N-1) x}{N}} I_2 s_2^2   N+200 e^{\frac{s_2 (N-2) x}{N}} s_1   s_2 N-200 e^{\frac{s_2 (N-2)   x}{N}+\frac{(1-I_2) s_2 (N-1)   x}{N}} s_1 s_2 N+200 e^{\frac{s_2   (N-2) x}{N}} I_2 s_1 s_2 N+100   e^{\frac{s_2 (N-2) x}{N}+\frac{(1-I_2)   s_2 (N-1) x}{N}} I_2 s_1 s_2   N-1 e^{\frac{(1-I_2) s_2 (N-1)   x}{N}} s_1^2 T_D N-1 e^{\frac{s_2   (N-2) x}{N}+\frac{(1-I_2) s_2   (N-1) x}{N}} s_1^2 T_D N+2   e^{\frac{s_2 (N-2) x}{N}+\frac{(1-I_2)   s_2 (N-1) x}{N}} s_1 s_2 T_D   N+100 e^{\frac{(1-I_2) s_2 (N-1)   x}{N}} s_1^2+100 e^{\frac{s_2 (N-2)   x}{N}+\frac{(1-I_2) s_2 (N-1)   x}{N}} s_1^2-200 e^{\frac{s_2 (N-2)   x}{N}+\frac{(1-I_2) s_2 (N-1)   x}{N}} s_1 s_2 \pm \sqrt{\left(-100 e^{\frac{s_2   (N-2) x}{N}+\frac{(1-I_2) s_2   (N-1) x}{N}} s_2^2 N^3+100   e^{\frac{s_2 (N-2) x}{N}+\frac{(1-I_2)   s_2 (N-1) x}{N}} I_2 s_2^2   N^3-100 e^{\frac{s_2 (N-2) x}{N}} I_2   s_1 s_2 N^3-100 e^{\frac{s_2 (N-2)   x}{N}} I_2 s_1^2 N^2+300 e^{\frac{s_2   (N-2) x}{N}+\frac{(1-I_2) s_2   (N-1) x}{N}} s_2^2 N^2-300   e^{\frac{s_2 (N-2) x}{N}+\frac{(1-I_2)   s_2 (N-1) x}{N}} I_2 s_2^2   N^2+100 e^{\frac{s_2 (N-2) x}{N}} s_1   s_2 N^2-200 e^{\frac{s_2 (N-2)   x}{N}+\frac{(1-I_2) s_2 (N-1)   x}{N}} s_1 s_2 N^2+300 e^{\frac{s_2   (N-2) x}{N}} I_2 s_1 s_2   N^2+100 e^{\frac{s_2 (N-2) x}{N}+\frac{(1-I_2) s_2 (N-1) x}{N}} I_2 s_1   s_2 N^2-1 e^{\frac{(1-I_2) s_2 (N-1)   x}{N}} s_1^2 T_D N^2+1 e^{\frac{s_2   (N-2) x}{N}+\frac{(1-I_2) s_2   (N-1) x}{N}} s_1 s_2 T_D   N^2+100 e^{\frac{s_2 (N-2) x}{N}}   s_1^2 N+100 e^{\frac{(1-I_2) s_2   (N-1) x}{N}} s_1^2 N-100 e^{\frac{s_2   (N-2) x}{N}+\frac{(1-I_2) s_2   (N-1) x}{N}} s_1^2 N+100 e^{\frac{s_2   (N-2) x}{N}} I_2 s_1^2 N-200   e^{\frac{s_2 (N-2) x}{N}+\frac{(1-I_2)   s_2 (N-1) x}{N}} s_2^2 N+200   e^{\frac{s_2 (N-2) x}{N}+\frac{(1-I_2)   s_2 (N-1) x}{N}} I_2 s_2^2   N-200 e^{\frac{s_2 (N-2) x}{N}} s_1   s_2 N+200 e^{\frac{s_2 (N-2)   x}{N}+\frac{(1-I_2) s_2 (N-1)   x}{N}} s_1 s_2 N-200 e^{\frac{s_2   (N-2) x}{N}} I_2 s_1 s_2 N-100   e^{\frac{s_2 (N-2) x}{N}+\frac{(1-I_2)   s_2 (N-1) x}{N}} I_2 s_1 s_2   N+1 e^{\frac{(1-I_2) s_2 (N-1)   x}{N}} s_1^2 T_D N+1 e^{\frac{s_2   (N-2) x}{N}+\frac{(1-I_2) s_2   (N-1) x}{N}} s_1^2 T_D N-2   e^{\frac{s_2 (N-2) x}{N}+\frac{(1-I_2)   s_2 (N-1) x}{N}} s_1 s_2 T_D   N-100 e^{\frac{(1-I_2) s_2 (N-1)   x}{N}} s_1^2-100 e^{\frac{s_2 (N-2)   x}{N}+\frac{(1-I_2) s_2 (N-1)   x}{N}} s_1^2+200 e^{\frac{s_2 (N-2)   x}{N}+\frac{(1-I_2) s_2 (N-1)   x}{N}} s_1 s_2\right)^2-4. s_1 N \left(-100   e^{\frac{s_2 (N-2) x}{N}} s_1+100   e^{\frac{s_2 (N-2) x}{N}+\frac{(1-I_2)   s_2 (N-1) x}{N}} s_1+200 e^{\frac{s_2   (N-2) x}{N}} s_2-200 e^{\frac{s_2   (N-2) x}{N}+\frac{(1-I_2) s_2   (N-1) x}{N}} s_2-100 e^{\frac{s_2   (N-2) x}{N}} s_2 N+100 e^{\frac{s_2   (N-2) x}{N}+\frac{(1-I_2) s_2   (N-1) x}{N}} s_2 N\right) \left(100   e^{\frac{s_2 (N-2) x}{N}} I_2 s_1 s_2   N^3-1 e^{\frac{s_2 (N-2) x}{N}+\frac{(1-I_2) s_2 (N-1) x}{N}} s_2^2 T_D   N^3+1 e^{\frac{s_2 (N-2) x}{N}+\frac{(1-I_2) s_2 (N-1) x}{N}} I_2 s_2^2   T_D N^3+1 e^{\frac{(1-I_2) s_2 (N-1)   x}{N}} s_1 s_2 T_D N^3-1 e^{\frac{(1 -1   I_2) s_2 (N-1) x}{N}} I_2 s_1   s_2 T_D N^3+100 e^{\frac{s_2 (N-2)   x}{N}} I_2 s_1^2 N^2+100 e^{\frac{s_2   (N-2) x}{N}+\frac{(1-I_2) s_2   (N-1) x}{N}} s_2^2 N^2-100   e^{\frac{s_2 (N-2) x}{N}+\frac{(1-I_2)   s_2 (N-1) x}{N}} I_2 s_2^2   N^2-100 e^{\frac{(1-I_2) s_2 (N-1)   x}{N}} s_1 s_2 N^2-300 e^{\frac{s_2   (N-2) x}{N}} I_2 s_1 s_2   N^2+100 e^{\frac{(1-I_2) s_2 (N-1)   x}{N}} I_2 s_1 s_2 N^2+1 e^{\frac{(1 -1   I_2) s_2 (N-1) x}{N}} s_1^2 T_D   N^2+3. e^{\frac{s_2 (N-2) x}{N}+\frac{(1-I_2) s_2 (N-1) x}{N}} s_2^2 T_D   N^2-3. e^{\frac{s_2 (N-2) x}{N}+\frac{(1-I_2) s_2 (N-1) x}{N}} I_2 s_2^2   T_D N^2-2 e^{\frac{(1-I_2) s_2 (N-1)   x}{N}} s_1 s_2 T_D N^2-2 e^{\frac{s_2   (N-2) x}{N}+\frac{(1-I_2) s_2   (N-1) x}{N}} s_1 s_2 T_D N^2+2   e^{\frac{(1-I_2) s_2 (N-1) x}{N}} I_2   s_1 s_2 T_D N^2+1 e^{\frac{s_2 (N-2)   x}{N}+\frac{(1-I_2) s_2 (N-1)   x}{N}} I_2 s_1 s_2 T_D N^2-100   e^{\frac{(1-I_2) s_2 (N-1) x}{N}}   s_1^2 N-100 e^{\frac{s_2 (N-2) x}{N}}   I_2 s_1^2 N-300 e^{\frac{s_2 (N-2)   x}{N}+\frac{(1-I_2) s_2 (N-1)   x}{N}} s_2^2 N+300 e^{\frac{s_2 (N-2)   x}{N}+\frac{(1-I_2) s_2 (N-1)   x}{N}} I_2 s_2^2 N+200 e^{\frac{(1 -1   I_2) s_2 (N-1) x}{N}} s_1 s_2   N+200 e^{\frac{s_2 (N-2) x}{N}+\frac{(1-I_2) s_2 (N-1) x}{N}} s_1 s_2   N+200 e^{\frac{s_2 (N-2) x}{N}} I_2   s_1 s_2 N-200 e^{\frac{(1-I_2) s_2   (N-1) x}{N}} I_2 s_1 s_2 N-100   e^{\frac{s_2 (N-2) x}{N}+\frac{(1-I_2)   s_2 (N-1) x}{N}} I_2 s_1 s_2   N-1 e^{\frac{(1-I_2) s_2 (N-1)   x}{N}} s_1^2 T_D N-1 e^{\frac{s_2   (N-2) x}{N}+\frac{(1-I_2) s_2   (N-1) x}{N}} s_1^2 T_D N-2   e^{\frac{s_2 (N-2) x}{N}+\frac{(1-I_2)   s_2 (N-1) x}{N}} s_2^2 T_D N+2   e^{\frac{s_2 (N-2) x}{N}+\frac{(1-I_2)   s_2 (N-1) x}{N}} I_2 s_2^2 T_D   N+1 e^{\frac{(1-I_2) s_2 (N-1)   x}{N}} s_1 s_2 T_D N+3. e^{\frac{s_2   (N-2) x}{N}+\frac{(1-I_2) s_2   (N-1) x}{N}} s_1 s_2 T_D N-1   e^{\frac{(1-I_2) s_2 (N-1) x}{N}} I_2   s_1 s_2 T_D N-1 e^{\frac{s_2 (N-2)   x}{N}+\frac{(1-I_2) s_2 (N-1)   x}{N}} I_2 s_1 s_2 T_D N+100   e^{\frac{(1-I_2) s_2 (N-1) x}{N}}   s_1^2+100 e^{\frac{s_2 (N-2) x}{N}+\frac{(1-I_2) s_2 (N-1) x}{N}} s_1^2+200   e^{\frac{s_2 (N-2) x}{N}+\frac{(1-I_2)   s_2 (N-1) x}{N}} s_2^2-200 e^{\frac{s_2   (N-2) x}{N}+\frac{(1-I_2) s_2   (N-1) x}{N}} I_2 s_2^2-100 e^{\frac{(1 -1   I_2) s_2 (N-1) x}{N}} s_1 s_2-300   e^{\frac{s_2 (N-2) x}{N}+\frac{(1-I_2)   s_2 (N-1) x}{N}} s_1 s_2+100 e^{\frac{(1  - I_2) s_2 (N-1) x}{N}} I_2 s_1  s_2+100 e^{\frac{s_2 (N-2) x}{N}+\frac{(1 -1   I_2) s_2 (N-1) x}{N}} I_2 s_1   s_2\right)}\right)}{\splitfrac{s_1 N \left(-100 e^{\frac{s_2   (N-2) x}{N}} s_1+100 e^{\frac{s_2   (N-2) x}{N}+\frac{(1-I_2) s_2   (N-1) x}{N}} s_1+200 e^{\frac{s_2   (N-2) x}{N}} s_2 \right. }{\splitfrac{-200 e^{\frac{s_2   (N-2) x}{N}+\frac{(1-I_2) s_2   (N-1) x}{N}} s_2-100 e^{\frac{s_2   (N-2) x}{N}} s_2 N}{\left. +100 e^{\frac{s_2   (N-2) x}{N}+\frac{(1-I_2) s_2 (N-1) x}{N}} s_2 N\right)}}}
 \end{align*}
 Similarly, the solution for $I_1$ when $\text{AverageChange}(N-1)=0$ is given by
 \begin{align*}
 UBR(I_2)=& \frac{-e^{\frac{(1-I_2) s_2 x}{N}} s_1^2   N^3+e^{\frac{(1-I_2) s_2 x}{N}+\frac{s_2   x}{N}} s_1^2 N^3+\frac{100 e^{\frac{s_2   x}{N}} s_1^2 N^3}{T_D}-\frac{100   e^{\frac{(1-I_2) s_2 x}{N}+\frac{s_2 x}{N}}   s_1^2 N^3}{T_D}+4 e^{\frac{(1-I_2) s_2   x}{N}} s_1^2 N^2-5 e^{\frac{(1-I_2) s_2   x}{N}+\frac{s_2 x}{N}} s_1^2   N^2+e^{\frac{(1-I_2) s_2 x}{N}+\frac{s_2   x}{N}} s_1 s_2 N^2-\frac{400 e^{\frac{s_2   x}{N}} s_1^2 N^2}{T_D}+\frac{100   e^{\frac{(1-I_2) s_2 x}{N}} s_1^2   N^2}{T_D}+\frac{300 e^{\frac{(1-I_2) s_2   x}{N}+\frac{s_2 x}{N}} s_1^2   N^2}{T_D}-\frac{100 e^{\frac{s_2 x}{N}} I_2   s_1^2 N^2}{T_D}+\frac{100 e^{\frac{s_2   x}{N}} s_1 s_2 N^2}{T_D}-\frac{200   e^{\frac{(1-I_2) s_2 x}{N}+\frac{s_2 x}{N}}   s_1 s_2 N^2}{T_D}+\frac{100 e^{\frac{(1-I_2)   s_2 x}{N}+\frac{s_2 x}{N}} I_2 s_1   s_2 N^2}{T_D}-5 e^{\frac{(1-I_2) s_2   x}{N}} s_1^2 N+8 e^{\frac{(1-I_2) s_2   x}{N}+\frac{s_2 x}{N}} s_1^2 N-3   e^{\frac{(1-I_2) s_2 x}{N}+\frac{s_2 x}{N}}   s_1 s_2 N+\frac{500 e^{\frac{s_2 x}{N}}   s_1^2 N}{T_D}-\frac{400 e^{\frac{(1-I_2) s_2   x}{N}} s_1^2 N}{T_D}+\frac{300 e^{\frac{s_2   x}{N}} I_2 s_1^2 N}{T_D}-\frac{100   e^{\frac{(1-I_2) s_2 x}{N}+\frac{s_2 x}{N}}   s_2^2 N}{T_D}+\frac{100 e^{\frac{(1-I_2) s_2   x}{N}+\frac{s_2 x}{N}} I_2 s_2^2   N}{T_D}-\frac{200 e^{\frac{s_2 x}{N}} s_1   s_2 N}{T_D}+\frac{400 e^{\frac{(1-I_2) s_2   x}{N}+\frac{s_2 x}{N}} s_1 s_2   N}{T_D}-\frac{100 e^{\frac{s_2 x}{N}} I_2   s_1 s_2 N}{T_D}-\frac{300 e^{\frac{(1-I_2)   s_2 x}{N}+\frac{s_2 x}{N}} I_2 s_1   s_2 N}{T_D}+2 e^{\frac{(1-I_2) s_2   x}{N}} s_1^2-4 e^{\frac{(1-I_2) s_2   x}{N}+\frac{s_2 x}{N}} s_1^2+2   e^{\frac{(1-I_2) s_2 x}{N}+\frac{s_2 x}{N}}   s_1 s_2 \pm \sqrt{\left(e^{\frac{(1-I_2) s_2 x}{N}}   s_1^2 N^3-e^{\frac{(1-I_2) s_2   x}{N}+\frac{s_2 x}{N}} s_1^2   N^3-\frac{100 e^{\frac{s_2 x}{N}} s_1^2   N^3}{T_D}+\frac{100 e^{\frac{(1-I_2) s_2   x}{N}+\frac{s_2 x}{N}} s_1^2   N^3}{T_D}-4 e^{\frac{(1-I_2) s_2 x}{N}}   s_1^2 N^2+5 e^{\frac{(1-I_2) s_2   x}{N}+\frac{s_2 x}{N}} s_1^2   N^2-e^{\frac{(1-I_2) s_2 x}{N}+\frac{s_2   x}{N}} s_1 s_2 N^2+\frac{400 e^{\frac{s_2   x}{N}} s_1^2 N^2}{T_D}-\frac{100   e^{\frac{(1-I_2) s_2 x}{N}} s_1^2   N^2}{T_D}-\frac{300 e^{\frac{(1-I_2) s_2   x}{N}+\frac{s_2 x}{N}} s_1^2   N^2}{T_D}+\frac{100 e^{\frac{s_2 x}{N}} I_2   s_1^2 N^2}{T_D}-\frac{100 e^{\frac{s_2   x}{N}} s_1 s_2 N^2}{T_D}+\frac{200   e^{\frac{(1-I_2) s_2 x}{N}+\frac{s_2 x}{N}}   s_1 s_2 N^2}{T_D}-\frac{100 e^{\frac{(1-I_2)   s_2 x}{N}+\frac{s_2 x}{N}} I_2 s_1   s_2 N^2}{T_D}+5 e^{\frac{(1-I_2) s_2   x}{N}} s_1^2 N-8 e^{\frac{(1-I_2) s_2   x}{N}+\frac{s_2 x}{N}} s_1^2 N+3   e^{\frac{(1-I_2) s_2 x}{N}+\frac{s_2 x}{N}}   s_1 s_2 N-\frac{500 e^{\frac{s_2 x}{N}}   s_1^2 N}{T_D}+\frac{400 e^{\frac{(1-I_2) s_2   x}{N}} s_1^2 N}{T_D}-\frac{300 e^{\frac{s_2   x}{N}} I_2 s_1^2 N}{T_D}+\frac{100   e^{\frac{(1-I_2) s_2 x}{N}+\frac{s_2 x}{N}}   s_2^2 N}{T_D}-\frac{100 e^{\frac{(1-I_2) s_2   x}{N}+\frac{s_2 x}{N}} I_2 s_2^2   N}{T_D}+\frac{200 e^{\frac{s_2 x}{N}} s_1   s_2 N}{T_D}-\frac{400 e^{\frac{(1-I_2) s_2   x}{N}+\frac{s_2 x}{N}} s_1 s_2   N}{T_D}+\frac{100 e^{\frac{s_2 x}{N}} I_2   s_1 s_2 N}{T_D}+\frac{300 e^{\frac{(1-I_2)   s_2 x}{N}+\frac{s_2 x}{N}} I_2 s_1   s_2 N}{T_D}-2 e^{\frac{(1-I_2) s_2   x}{N}} s_1^2+4 e^{\frac{(1-I_2) s_2   x}{N}+\frac{s_2 x}{N}} s_1^2-2   e^{\frac{(1-I_2) s_2 x}{N}+\frac{s_2 x}{N}}   s_1 s_2+\frac{200 e^{\frac{s_2 x}{N}}   s_1^2}{T_D}-\frac{500 e^{\frac{(1-I_2) s_2 x}{N}}   s_1^2}{T_D}+\frac{600 e^{\frac{(1-I_2) s_2   x}{N}+\frac{s_2 x}{N}} s_1^2}{T_D}+\frac{200   e^{\frac{s_2 x}{N}} I_2 s_1^2}{T_D}-\frac{100   e^{\frac{(1-I_2) s_2 x}{N}+\frac{s_2 x}{N}}   s_2^2}{T_D}+\frac{100 e^{\frac{(1-I_2) s_2   x}{N}+\frac{s_2 x}{N}} I_2   s_2^2}{T_D}-\frac{100 e^{\frac{s_2 x}{N}} s_1   s_2}{T_D}-\frac{100 e^{\frac{s_2 x}{N}} I_2   s_1 s_2}{T_D}-\frac{200 e^{\frac{(1-I_2) s_2   x}{N}+\frac{s_2 x}{N}} I_2 s_1   s_2}{T_D}+\frac{200 e^{\frac{(1-I_2) s_2 x}{N}}   s_1^2}{T_D N}-\frac{400 e^{\frac{(1-I_2) s_2   x}{N}+\frac{s_2 x}{N}} s_1^2}{T_D   N}+\frac{200 e^{\frac{(1-I_2) s_2   x}{N}+\frac{s_2 x}{N}} s_1 s_2}{T_D   N}\right)^2-4 \left(-e^{\frac{(1-I_2) s_2 x}{N}}   s_1^2 N^3+e^{\frac{(1-I_2) s_2   x}{N}+\frac{s_2 x}{N}} s_1^2 N^3+4   e^{\frac{(1-I_2) s_2 x}{N}} s_1^2 N^2-5   e^{\frac{(1-I_2) s_2 x}{N}+\frac{s_2 x}{N}}   s_1^2 N^2-e^{\frac{(1-I_2) s_2 x}{N}}   s_1 s_2 N^2+2 e^{\frac{(1-I_2) s_2   x}{N}+\frac{s_2 x}{N}} s_1 s_2   N^2+e^{\frac{(1-I_2) s_2 x}{N}} I_2   s_1 s_2 N^2-e^{\frac{(1-I_2) s_2   x}{N}+\frac{s_2 x}{N}} I_2 s_1 s_2   N^2+\frac{100 e^{\frac{(1-I_2) s_2 x}{N}}   s_1^2 N^2}{T_D}-\frac{100 e^{\frac{(1-I_2) s_2   x}{N}+\frac{s_2 x}{N}} s_1^2   N^2}{T_D}-\frac{100 e^{\frac{s_2 x}{N}} I_2   s_1^2 N^2}{T_D}-5 e^{\frac{(1-I_2) s_2   x}{N}} s_1^2 N+8 e^{\frac{(1-I_2) s_2   x}{N}+\frac{s_2 x}{N}} s_1^2   N+e^{\frac{(1-I_2) s_2 x}{N}+\frac{s_2   x}{N}} s_2^2 N-e^{\frac{(1-I_2) s_2   x}{N}+\frac{s_2 x}{N}} I_2 s_2^2   N+3 e^{\frac{(1-I_2) s_2 x}{N}} s_1   s_2 N-7 e^{\frac{(1-I_2) s_2   x}{N}+\frac{s_2 x}{N}} s_1 s_2   N-3 e^{\frac{(1-I_2) s_2 x}{N}} I_2   s_1 s_2 N+4 e^{\frac{(1-I_2) s_2   x}{N}+\frac{s_2 x}{N}} I_2 s_1 s_2   N-\frac{400 e^{\frac{(1-I_2) s_2 x}{N}}   s_1^2 N}{T_D}+\frac{500 e^{\frac{(1-I_2) s_2   x}{N}+\frac{s_2 x}{N}} s_1^2   N}{T_D}+\frac{300 e^{\frac{s_2 x}{N}} I_2   s_1^2 N}{T_D}+\frac{100 e^{\frac{(1-I_2) s_2   x}{N}} s_1 s_2 N}{T_D}-\frac{200   e^{\frac{(1-I_2) s_2 x}{N}+\frac{s_2 x}{N}}   s_1 s_2 N}{T_D}-\frac{100 e^{\frac{s_2   x}{N}} I_2 s_1 s_2 N}{T_D}-\frac{100   e^{\frac{(1-I_2) s_2 x}{N}} I_2 s_1 s_2   N}{T_D}+\frac{100 e^{\frac{(1-I_2) s_2   x}{N}+\frac{s_2 x}{N}} I_2 s_1 s_2   N}{T_D}+2 e^{\frac{(1-I_2) s_2 x}{N}}   s_1^2-4 e^{\frac{(1-I_2) s_2 x}{N}+\frac{s_2   x}{N}} s_1^2-2 e^{\frac{(1-I_2) s_2   x}{N}+\frac{s_2 x}{N}} s_2^2+2   e^{\frac{(1-I_2) s_2 x}{N}+\frac{s_2 x}{N}}   I_2 s_2^2-2 e^{\frac{(1-I_2) s_2 x}{N}} s_1   s_2+6 e^{\frac{(1-I_2) s_2 x}{N}+\frac{s_2   x}{N}} s_1 s_2+2 e^{\frac{(1-I_2) s_2   x}{N}} I_2 s_1 s_2-4 e^{\frac{(1-I_2) s_2   x}{N}+\frac{s_2 x}{N}} I_2 s_1   s_2+\frac{500 e^{\frac{(1-I_2) s_2 x}{N}}  s_1^2}{T_D}-\frac{800 e^{\frac{(1-I_2) s_2   x}{N}+\frac{s_2 x}{N}} s_1^2}{T_D}-\frac{200   e^{\frac{s_2 x}{N}} I_2 s_1^2}{T_D}-\frac{100   e^{\frac{(1-I_2) s_2 x}{N}+\frac{s_2 x}{N}}   s_2^2}{T_D}+\frac{100 e^{\frac{(1-I_2) s_2   x}{N}+\frac{s_2 x}{N}} I_2   s_2^2}{T_D}-\frac{300 e^{\frac{(1-I_2) s_2 x}{N}}   s_1 s_2}{T_D}+\frac{700 e^{\frac{(1-I_2) s_2   x}{N}+\frac{s_2 x}{N}} s_1   s_2}{T_D}+\frac{100 e^{\frac{s_2 x}{N}} I_2   s_1 s_2}{T_D}+\frac{300 e^{\frac{(1-I_2) s_2   x}{N}} I_2 s_1 s_2}{T_D}-\frac{400   e^{\frac{(1-I_2) s_2 x}{N}+\frac{s_2 x}{N}}   I_2 s_1 s_2}{T_D}-\frac{200 e^{\frac{(1-I_2) s_2   x}{N}} s_1^2}{T_D N}+\frac{400   e^{\frac{(1-I_2) s_2 x}{N}+\frac{s_2 x}{N}}   s_1^2}{T_D N}+\frac{200 e^{\frac{(1-I_2) s_2   x}{N}+\frac{s_2 x}{N}} s_2^2}{T_D   N}-\frac{200 e^{\frac{(1-I_2) s_2   x}{N}+\frac{s_2 x}{N}} I_2 s_2^2}{T_D   N}+\frac{200 e^{\frac{(1-I_2) s_2 x}{N}} s_1   s_2}{T_D N}-\frac{600 e^{\frac{(1-I_2) s_2   x}{N}+\frac{s_2 x}{N}} s_1 s_2}{T_D   N}-\frac{200 e^{\frac{(1-I_2) s_2 x}{N}} I_2   s_1 s_2}{T_D N}+\frac{400 e^{\frac{(1-I_2)   s_2 x}{N}+\frac{s_2 x}{N}} I_2 s_1   s_2}{T_D N}\right) \left(\frac{100 e^{\frac{s_2   x}{N}} s_1^2 N^3}{T_D}-\frac{100   e^{\frac{(1-I_2) s_2 x}{N}+\frac{s_2 x}{N}}   s_1^2 N^3}{T_D}-\frac{400 e^{\frac{s_2   x}{N}} s_1^2 N^2}{T_D}+\frac{400   e^{\frac{(1-I_2) s_2 x}{N}+\frac{s_2 x}{N}}   s_1^2 N^2}{T_D}+\frac{100 e^{\frac{s_2   x}{N}} s_1 s_2 N^2}{T_D}-\frac{100   e^{\frac{(1-I_2) s_2 x}{N}+\frac{s_2 x}{N}}   s_1 s_2 N^2}{T_D}+\frac{500 e^{\frac{s_2   x}{N}} s_1^2 N}{T_D}-\frac{500   e^{\frac{(1-I_2) s_2 x}{N}+\frac{s_2 x}{N}}   s_1^2 N}{T_D}-\frac{200 e^{\frac{s_2 x}{N}}   s_1 s_2 N}{T_D}+\frac{200 e^{\frac{(1-I_2)   s_2 x}{N}+\frac{s_2 x}{N}} s_1 s_2   N}{T_D}-\frac{200 e^{\frac{s_2 x}{N}}   s_1^2}{T_D}+\frac{200 e^{\frac{(1-I_2) s_2   x}{N}+\frac{s_2 x}{N}} s_1^2}{T_D}+\frac{100   e^{\frac{s_2 x}{N}} s_1 s_2}{T_D}-\frac{100   e^{\frac{(1-I_2) s_2 x}{N}+\frac{s_2 x}{N}}   s_1 s_2}{T_D}\right)}-\frac{200 e^{\frac{s_2   x}{N}} s_1^2}{T_D}+\frac{500 e^{\frac{(1-I_2) s_2   x}{N}} s_1^2}{T_D}-\frac{600 e^{\frac{(1-I_2) s_2   x}{N}+\frac{s_2 x}{N}} s_1^2}{T_D}-\frac{200   e^{\frac{s_2 x}{N}} I_2 s_1^2}{T_D}+\frac{100   e^{\frac{(1-I_2) s_2 x}{N}+\frac{s_2 x}{N}}   s_2^2}{T_D}-\frac{100 e^{\frac{(1-I_2) s_2   x}{N}+\frac{s_2 x}{N}} I_2   s_2^2}{T_D}+\frac{100 e^{\frac{s_2 x}{N}} s_1   s_2}{T_D}+\frac{100 e^{\frac{s_2 x}{N}} I_2   s_1 s_2}{T_D}+\frac{200 e^{\frac{(1-I_2) s_2   x}{N}+\frac{s_2 x}{N}} I_2 s_1   s_2}{T_D}-\frac{200 e^{\frac{(1-I_2) s_2 x}{N}}   s_1^2}{T_D N}+\frac{400 e^{\frac{(1-I_2) s_2   x}{N}+\frac{s_2 x}{N}} s_1^2}{T_D   N}-\frac{200 e^{\frac{(1-I_2) s_2   x}{N}+\frac{s_2 x}{N}} s_1 s_2}{T_D   N}}{\splitfrac{2 \left(\frac{100 e^{\frac{s_2 x}{N}} s_1^2   N^3}{T_D}-\frac{100 e^{\frac{(1-I_2) s_2   x}{N}+\frac{s_2 x}{N}} s_1^2   N^3}{T_D}-\frac{400 e^{\frac{s_2 x}{N}}   s_1^2 N^2}{T_D}+\frac{400 e^{\frac{(1-I_2) s_2   x}{N}+\frac{s_2 x}{N}} s_1^2   N^2}{T_D}\right.}{ \splitfrac{+\frac{100 e^{\frac{s_2 x}{N}} s_1   s_2 N^2}{T_D}-\frac{100 e^{\frac{(1-I_2) s_2   x}{N}+\frac{s_2 x}{N}} s_1 s_2   N^2}{T_D}+\frac{500 e^{\frac{s_2 x}{N}}   s_1^2 N}{T_D}-\frac{500 e^{\frac{(1-I_2) s_2   x}{N}+\frac{s_2 x}{N}} s_1^2   N}{T_D}}{\splitfrac{-\frac{200 e^{\frac{s_2 x}{N}} s_1   s_2 N}{T_D}+\frac{200 e^{\frac{(1-I_2) s_2   x}{N}+\frac{s_2 x}{N}} s_1 s_2   N}{T_D}-\frac{200 e^{\frac{s_2 x}{N}}   s_1^2}{T_D}+\frac{200 e^{\frac{(1-I_2) s_2   x}{N}+\frac{s_2 x}{N}} s_1^2}{T_D}}{\left.+\frac{100   e^{\frac{s_2 x}{N}} s_1 s_2}{T_D}-\frac{100   e^{\frac{(1-I_2) s_2 x}{N}+\frac{s_2 x}{N}} s_1 s_2}{T_D}\right)}}}} \end{align*}
 Note that for both the growth-defence trade-off, and the full three dimensional trade-off that also includes fecundity, one of the roots here is always above one. These roots are the positive square root for $LBR(I_2)$, and the negative root for $UBR(I_2)$. Therefore, the roots used to calculate the size of the region of coexistence are the negative and positive roots respectively.
 
 \begin{equation}
\frac{1}{\splitfrac{2 \left(\frac{100 e^{\frac{s_2 x}{N}} s_1^2   N^3}{T_D}-\frac{100 e^{\frac{(1-I_2) s_2   x}{N}+\frac{s_2 x}{N}} s_1^2   N^3}{T_D}-\frac{400 e^{\frac{s_2 x}{N}}   s_1^2 N^2}{T_D}+\frac{400 e^{\frac{(1-I_2) s_2   x}{N}+\frac{s_2 x}{N}} s_1^2   N^2}{T_D}\right.}{ \splitfrac{+\frac{100 e^{\frac{s_2 x}{N}} s_1   s_2 N^2}{T_D}-\frac{100 e^{\frac{(1-I_2) s_2   x}{N}+\frac{s_2 x}{N}} s_1 s_2   N^2}{T_D}+\frac{500 e^{\frac{s_2 x}{N}}   s_1^2 N}{T_D}-\frac{500 e^{\frac{(1-I_2) s_2   x}{N}+\frac{s_2 x}{N}} s_1^2   N}{T_D}}{\splitfrac{-\frac{200 e^{\frac{s_2 x}{N}} s_1   s_2 N}{T_D}+\frac{200 e^{\frac{(1-I_2) s_2   x}{N}+\frac{s_2 x}{N}} s_1 s_2   N}{T_D}-\frac{200 e^{\frac{s_2 x}{N}}   s_1^2}{T_D}+\frac{200 e^{\frac{(1-I_2) s_2   x}{N}+\frac{s_2 x}{N}} s_1^2}{T_D}}{\left.+\frac{100   e^{\frac{s_2 x}{N}} s_1 s_2}{T_D}-\frac{100   e^{\frac{(1-I_2) s_2 x}{N}+\frac{s_2 x}{N}} s_1 s_2}{T_D}\right)}}}}
\end{equation}

\bibliographystyle{plos}
\bibliography{Papers.bib}
