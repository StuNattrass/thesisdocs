Tropical forests and other communities often exhibit very high species richness \citep[e.g.][]{gentry1988tree,valencia1994high,walter1971ecology}, and understanding the mechanisms driving this diversity remains a crucial problem within ecology. This coexistence does not fit the classical explanation, where each species occupies it's own niche, as most plant species require the same resources and, and use similar methods to gather these resources \citep{silvertown2004plant}. This thesis has focussed on the effects of competitive asymmetry and disturbance events on species richness.

In Chapter~1, we begin by determining the effect of competitive asymmetry on the likelihood of species coexisting. Specifically, we used a Lotka-Volterra type model to investigate how the asymmetry between very similar species affects the probability of coexistence. We find that as the slope of the competition kernel at zero increases, such that competition between similar species becomes more asymmetric, the likelihood of randomly selected species coexisting is increased. Coexistence is dependent on the steepness of the competition kernel at the origin, and the probability of coexistence is maximised when the competition kernel is described by a step function, and the slope of the kernel is infinite. Further, the probability of coexistence increases with the maximum degree of asymmetry between species, highlighting the importance of competitive asymmetry in sustaining biodiversity. However, even with the trade-off assumptions that maximise coexistence, we find that the likelihood of coexistence decreases as more species are introduced into the system. This highlights that while asymmetry is an important driver of diversity, it is expected to interact with other factors to sustain the highly diverse communities observed in nature.

In Chapter~2, we use the results of Chapter~1 to inform a more mechanistic, individual based model of forest gap replenishment. Competitive asymmetry is included in the form of juvenile competition for light, where by the tallest individual will always outcompete a smaller individual for a place in the canopy. We use a lottery type model \citep{sale1978coexistence,chesson1981environmental} to simulate gap replacement in the canopy, where species differ in seed production (seed number per capita per annum) and juvenile growth rate. As in the Lotka-Volterra model used in Chapter~1, coexistence is only possible when there is a trade-off between these factors. We then extend this model to consider the effects of disturbance events on coexistence, and find that disturbance can dramatically increase the range of parameters for which coexistence can occur. An examination of the diversity-disturbance relationships (DDR) generated then occurs. One of the most significant hypotheses on the effects of disturbance on species richness is the intermediate disturbance hypothesis (IDH). The IDH proposes that  if disturbance is too low, late successional species, often species that produce fewer, larger seeds, will exclude more fecund pioneer species. However, when disturbance pressure is high, the pioneer species that can colonise empty sites effectively will exclude the less fecund, late successional species. At intermediate levels of disturbance, both pioneer and late-successional species are expected to coexist, resulting in a diversity peak at intermediate disturbance. However, this is not necessarily what we observe. While for some choices of parameters, we found that an increase in disturbance intensity will produce the humped DDR, other DDRs are also common. We argue that the IDH is therefore only one possible outcome, and that the effects of disturbance can vary dependent upon the way it is measured. Different factors that influence disturbance can  give very different responses, even though the average death rate is identical. We also demonstrate that these results are robust to changes in system capacity. the maximum number of individuals a community can sustain, and show that increases/decreases in productivity are equivalent to an increase/decrease in the difference between the juvenile growth rates of the species. We therefore argue that productivity and the life history of the species present may allow the prediction of disturbance impact on a community.

The model developed in Chapter~2 is broadened to consider different trade-offs in Chapter~3, where we analyse the probability of coexistence for a variety of trade-offs within the lottery model framework. We consider trade-offs that allow species to differ in disturbance defence, along with fecundity and juvenile growth, and aimed to determine analytically which of these trade-offs are the strongest drivers of diversity. We show that while trade-offs that include a variation in juvenile growth give coexistence that is robust to changes in system capacity, a trade-off between fecundity and disturbance defence is highly sensitive to changes in system capacity. Further, we show that coexistence is more likely for a fecundity-growth trade off than either a trade-off between disturbance defence and either fecundity or growth., and that including defence differences in a fecundity-growth trade-off does not significantly increase the likelihood of coexistence. We conclude that the fecundity-growth trade-off - a particular form of the competition-colonisation trade-off - is a significant driver of biodiversity, in accordance with previous studies and the results of Chapter~1 \citep[e.g.][]{adler2000space,nattrass2012quantifying,tilman1994competition}. Further, while there is limited empirical evidence for a fecundity-defence trade-off, this is an affect of other, more strongly selected trade-offs such as those between fecundity and growth, or growth and defence.

Finally, the lottery model is expanded to include a third species, and the conditions for greater than two species coexistence are examined in Chapter~4. We find that in a single, well mixed community, coexistence of all three species cannot occur. However, when species are organised using a strict fecundity-growth trade-off, if the community consists of two patches where dispersal between patches is limited, and the patches experience different disturbance pressures, regional coexistence of three species is possible. We then consider the effects of disturbance extent by varying the size of a patch with no disturbance alongside a patch where disturbance pressure is varied. We find that coexistence is more likely when the sheltered or protected region is small compared to the area experiencing disturbance. The implications of these results for human induced disturbance such as logging are discussed, and we find that the creation of nature reserves may boost regional diversity, even when diversity within the reserve itself does not increase. This highlights the importance of considering the effects of disturbance on multiple spatial scales, as the observed effects may differ if focus is too narrow, and only diversity within reserves is considered.

This thesis has investigated the effects of disturbance events and various trade-offs on species richness. In particular, we have examined  how different factors that determine the overall disturbance regime can have different effects on diversity, and shown combining these different disturbance factors can either boost or decrease diversity. The model presented in Chapter~2 and developed in Chapters~3~and~4 may prove a useful framework for future work on the effects of disturbance. This model attempts to reconcile aspects of niche theory, where the differences between species determine the outcome of competition  \citep[e.g.][]{darwinorigin,tilman1980resources,tilman1991dynamics,tilman1994competition}, and neutral theory, where all species are considered equivalent, and chance is the factor determining the long term dynamics of a community  \citep[e.g.][]{hubbell1979tree,chave2004neutral,hubbell2001unified}. Here, chance is included via stochastic dispersal of seeds, and can have a significant effect on the dynamics, while species are still distinguished by different life history traits. Hence, chance can lead one species (or more) to extinction even when invasion analysis suggests coexistence should occur. This is especially the case when the quasi-equilibria defined in Chapter~2 have one species maintained at a low population, close to 0. However, species do differ in life history traits, and the model includes three of those traits; fecundity as measured by seed production per capita, juvenile growth rates and defence against disturbance.

The separation of different factors influencing disturbance is a key step towards understanding the impact of disturbance events. The results in Chapter~2 show that each of these factors can have very different effects on diversity as disturbance regimes alter, with increases in intensity likely to produce humped DDRs, while a change in frequency is expected to produce an monotonic or flat DDR. This may help to explain the inconclusive results from empirical studies regarding the IDH, and echoes the insight of \cite{miller2011frequency}, who indicate that different disturbance regimes may give different results, even when the average death rate is identical. \cite{mackey2001diversity} reviewed the empirical evidence for the humped DDR predicted by the IDH, yet found the DDR was peaked in only 16 of 87 studies. They found that monotonically increasing or decreasing DDRs were almost as common as the peaked DDR, as were flat DDRs, where disturbance has no effect on diversity. In Chapter~2, we find that the model predicts that each of these DDR types is expected under certain parameter combinations. Both the empirical data and the model suggest that these DDRs are common, although U-shaped DDRs, where diversity in minimised as intermediate disturbance, are uncommon. The model developed here predicts that a U-shaped DDR will only occur for a narrow band of changes in disturbance, which involve a decrease from high to intermediate intensity, while higher frequency increases total area lost over a given time period. Mackey and Currie find only 3 studies which suggest a U-shaped DDR occurs.

Frequency and intensity, however, are not the only factors influencing the impact a disturbance event has. In Chapter~4, we expand the model to include an examination of disturbance extent. Through this, we illustrate that a combination of disturbance extent and intensity/frequency can increase the regional diversity. Further, this boost to diversity in the model occurs outside the protected, or reserve, region. Therefore, the species area relationship, whereby large areas of a habitat are expected to contain more species than a smaller region of the same habitat, suggests in the case modelled here, smaller regions that are protected from disturbance may be more effective at promoting region wide coexistence of multiple species \citep[e.g.][]{arrhenius1921species,gleason1922relation,connor1979statistics}. Indeed, the current model does predict this, where smaller relative reserve size can support higher regional diversity for a greater level of mixing between patches than a relatively large reserve region.

Further, we note the potential of this model framework extends beyond the results discussed in this thesis. At present, within each patch populations are assumed to be well mixed, although in reality, spatial structure within a community is expected to have an important role to play \citep{murrell2010does}. The current framework could be extended to include a true spatial structure, with the probability of site colonisation by a given species now dependent on the position of the patch and the spatial structure of the population. This extension  is the subject of planned future work. The introduction of this spatial structure would also allow for dispersal within and between patches to be governed by the same process, unlike in the current model, where separate mathematical rules govern intra- and inter-patch seed dispersal. Further, extending the model to include slow gap replenishment should also be possible, by introducing size structure such as that used in integral projection models \citep[e.g.][]{zuidema2010integral,rees2009integral}.

The mechanistic model considered in Chapters~2-4 only considered two or three species. One further extension of the model would be to include an increased number of species. While niche theory has considered coexistence of large communities, these studies are often phenomenological models without an explicit mechanism \citep[e.g. Chapter 1;][]{nattrass2012quantifying,kondoh2001unifying,adler2000space}. Even in these models, coexistence of large numbers of species is unlikely \citep[Chapter~1;][]{nattrass2012quantifying} or structurally unstable \citep{gyllenberg2005impossibility}, while competitive asymmetry is crucial to what coexistence can occur along a trade-off \citep[e.g. Chapter~1;][]{nattrass2012quantifying,adler2000space}. However, other hypotheses could be included in the current model, and combining these may give a mechanistic explanation for the coexistence of many species. \cite{tilman1982resource} suggests resource competition as a mechanism that may support multiple species, and the inclusion of competition for multiple resources (rather than light being considered as the only limiting resource as in the work presented here) may have a significant effect on the results when disturbance is considered. Previous work has also  suggested that predation or parasitism can enhance biodiversity \citep[see review in][]{chase2002interaction}, although this result may not be as strong as previously thought \citep{chesson2008interaction}. Adding multiple trophic levels into the model may explain greater levels of diversity, but at the expense of tractability. In addition to this effect, another possible driver of diversity is the Janzen-Connell hypothesis, which suggests host-specific predation or parasitism may make the region close to a parent individual inhospitable to its offspring, thus creating space for other species to persist \citep{janzen1970herbivores,connell1971role}, although the theoretical work on this hypothesis is not very mature at present. The model presented here attempts to combine aspects of niche theory and neutral theory. Neutral theory relies only on the balance between stochastic extinction and immigration/speciation to retain diversity. The work presented here illustrates that when species differ, as they are observed to in nature, immigration or speciation is not necessary for long term coexistence. However, neutral theory shows these to be a potentially important factor in sustaining diversity. While Chapter~4 demonstrates that immigration can lead to increased diversity on a local scale, we do not consider speciation, or evolution within a single species. We anticipate that including evolutionary dynamics will have complex effects, that may decrease or increase diversity as evolutionary forces interact with the ecological processes considered here.

To conclude, our investigation of the effects of disturbances has highlighted that simple hypotheses such as the IDH may not be adequate to describe the community response  to disturbance. Rather, we emphasise the need to consider that different factors influencing the ``total'' disturbance rate may have dramatically different effects on diversity, such as frequency, intensity and extent. In particular, we show that increases in frequency are unlikely to give the humped DDRs predicted by the IDH, while increases in intensity are more likely to match the predictions of the IDH. Further, we show that in some cases, moderating the extent of disturbance events by creating a sheltered region can improve regional diversity, although this increase may not be observed if only changes within the reserve are measured. We suggest that for a full mechanistic explanation of coexistence of multiple species, it may be necessary to consider models with multiple trade-offs. Despite this, much theory remains focussed on only two species, perhaps for tractability reasons \citep[e.g.][]{miller2011frequency,chesson2008interaction}. Building on the findings presented here to further enhance our theoretical understanding of disturbance remains a crucial part of ecology, with the potential to dramatically improve our management of the natural world.

