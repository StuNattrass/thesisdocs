
One of the key issues in ecology is understanding how some communities can maintain high levels of species diversity. In particular, tropical forests and coral reefs are known to support a large number of species, often within a small area. For example, a single hectare of Amazonian forest has been observed to support over 300 species (Valencia et al 1994), while a single region of coral reef ($<15$m$^3$) can host over 75 species (e.g. Smith and Taylor 1972). However, although the preservation of this diversity is a conservation priority (eg REFS), the processes that lead to, and consolidate, this biodiversity are little understood.

Much of the recent theoretical work on this topic has fallen into two categories; niche theory, where species specialise on different resources, and neutral theory, which considers all species ecologically equivalent and dynamics are determined by chance, either demographic or environmental stochasticity. Niche theory can be traced back as far as Darwin, who suggested a deterministic set of events resulting in different species each occupying a unique role within a community (Darwin 1859). More recently, Hutchinson noted the distinction between a species abiotic niche, the range of environments in which it can survive, and the realised niche, a function not only of abiotic factors but also dependent on the identity and actions of other species present in the community (Hutchinson 1958). Theory suggests that for a single resource, the species which can reduce this resource to the lowest level will exclude all other species ($R_0$ REFS), leading to the conclusion that for a given number $n$ of distinct resources, niche specialisation can result in $n$ species coexisting, as each maximises the use of a single resource, while the law of competitive exclusion (Hardin 1960) indicates that $m>n$ species cannot coexist. However, this mechanism often requires an excessive number of distinct resources in order to mirror the biodiversity levels in empirical studies.

 One possible explanation for diversity observed beyond the number of distinct resources is trade-off theory. Since resources are not infinite, no single species can allocate sufficient resource to optimise every life history trait, becoming Law's `Darwinian Demon' (Law 1979). Rather, resources are allocated to one life history trait  at the expense of others, resulting in trade-offs such that, for example, a species that produces seeds with a large mass will produce a much smaller number of seeds (eg Greene and Johnson, Venable), or a species with resource allocation to rapid juvenile growth may do so at the expense of its ability to disperse and colonise other areas in the environment (eg Tilman, Cadotte). Trade-offs such as the competition-colonisation trade-off have been demonstrated to theoretically support an arbitrarily large number of species (Tilman 1994).
 
 Trade-off theory may also be heavily dependent on competitive asymmetry between species. Competitive asymmetry occurs when one individual receives a disproportionate amount of the resource when in competition with another individual. Asymmetric competition is prevalent in insect communities (Lawton and Hassell 1981), and is also common within plants, where competition for light is expected to be dependent on size, with the larger individual intercepting large quantities of light at the expense of a smaller neighbour (Weiner 1990). Competition is said to be highly asymmetric if a small difference between individuals results in a large difference in the amount of resource garnered from competition.
 
 In Chapter 1, we examine the effects of this competitive asymmetry on the likelihood of coexistence. Using a Lotka-Volterra type model of competition, we find analytical conditions for the coexistence of two species along a fecundity-competition trade-off. We demonstrate that competitive asymmetry must be sufficiently high if the system is to support two or more species, and show that the likelihood of two species coexistence increases with competitive asymmetry, being maximised when competition is given by a discontinuous step function. Further, using this step function, we show analytically that this trade-off can support any number of species, although the probability of the system supporting randomly selected species declines rapidly as the number of species increases.
 
 While niche theory focuses on the differences between species, neutral theory instead considers all species equivalent at an individual level (Hubbell 1979), with community structure determined by chance. Designed as null models for coexistence, neutral theory models have proven able to predict high levels of biodiversity (REFS), and has accurately captured a range of observed relative abundance distributions (REFS). One type of model that utilises many of the assumptions of neutral theory are lottery models, introduced by Sale (1977) and developed further by Chesson (YR), where space is considered a limiting factors and the occupation of sites by a species is determined by the proportion of sites occupied by that species.

In Chapter 2, we develop a lottery type model in which we combine aspects of both neutral and niche theories. Using a trade-off between fecundity and juvenile growth rate, we introduce species specific characteristic into a stochastic model where  allocation of space is subject to significant environmental stochasticity. We demonstrate that this combination of species specific traits and chance can lead to coexistence of at least two species competing asymmetrically for light, and using stochastic boundedness as a criteria for coexistence, find express conditions for this coexistence. 

Further, we use this stochastic model to study the effects of disturbance on the community structure. Disturbance has been suggested as a strong driver of biodiversity, by perturbing the system such as to prevent the realisation of a lower diversity equilibrium situation (e.g. Sousa 1984, Denslow 1987). These events, defined by Shea et al. (2004) as events that alter niche opportunities via the death of large numbers of individuals, can create an environment where space limited species, such as pioneers, that prosper in an unsaturated environment, can persist in regions where they would otherwise by excluded if disturbance was too weak. If disturbance levels are sufficiently high, however, those species that are strong in saturated and congested environments will be excluded, as the environment will favour space limited species. Using these arguments, the intermediate disturbance hypothesis (IDH) suggests that diversity will be maximised at intermediate levels of disturbance, producing a unimodal, peaked diversity-disturbance relationship (DDR) (eg Connell 1978, Grime 1973, Huston 1979). However, empirical evidence for the IDH is mixed (see review in Mackey and Currie 2001), while it has also attracted theoretical criticism (e.g. Fox 2012). We expand our model to include disturbance events determined by two factors, frequency and intensity (proportion of individuals killed), and find that disturbance can dramatically increase the range of parameters for which two species can coexist. We demonstrate that the aspect of disturbance measured can have very different effects on the observed DDR, such that the peaked DDR predicted by the IDH is just one of several possible results from the same mechanism, and show that system capacity, the maximum number of individuals supported in the community, is crucial in determining how the community will react to disturbance.

While the focus of the work presented in Chapter 2 is a fecundity-growth trade-off, there are three main life history traits to consider for plants - fecundity, growth and defence against herbivory or other disturbance (REF). Therefore, in Chapter 3 we adapt the stochastic model in order to consider trade-offs between different combinations of these three traits, by introducing species specific disturbance intensities. We then compare the likelihood of these different trade-offs allowing two species coexistence, and compare the resulting predictions to empirical evidence from the literature. We find that using the lottery model framework, trade-offs including growth rate differences between species can support coexistence for any system capacity, while a trade-off between fecundity and defence cannot sustain more than a single species in large systems, and this is supported by the empirical results, showing no relationship between fecundity and defence, while growth-defence and fecundity growth trade-offs are observed in natural communities. By correlating the species specific intensities, we demonstrate that the fecundity-growth trade-off is likely to contribute the most to coexistence.

Using these conclusions, we then extend the stochastic model in Chapter 4 to include a third species along a fecundity-growth trade-off, and examine the effects of disturbance with greater than two species. Introducing a third species allows the consideration of coexistence between specialist species (with high fecundity or rapid juvenile growth rate) and a generalist with intermediate seed production and growth rates. We examine the effects of varied trade-offs and disturbance regimes, and introduce regions of the community that may be protected from disturbance, or experience a different disturbance regime from the remaining forest sites. We show that these refuges can provide for species that would otherwise be excluded if exposed to the conditions in the rest of the community, suggesting that disturbance regimes alone can allow native species to survive the invasion of an invasive species. Further, the model predicts that the effectiveness of nature reserves may be best observed in the regions neighbouring a reserve, rather than within the reserve itself. The potential effects of management strategies on species richness are then considered.

This thesis seeks to address the issues of determining both the type of competition best capable of sustaining a diverse community, and the predicted effects of disturbance regimes on species richness. These two processes aim to form preliminary steps towards a full mechanistic understanding of how diversity is maintained.

